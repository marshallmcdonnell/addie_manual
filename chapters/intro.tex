\chapter{Introduction}
\section{What is ADDIE} 
The ADvanced DIffraction Environment (ADDIE) is User software for 
reducing and analyzing data on the  
\href{http://https://neutrons.ornl.gov/nomad}{Nanoscale Ordered MAterials Diffractometer}  (NOMAD)  instrument at the \href{https://neutrons.ornl.gov/sns}{Spallation Neuron Source} (SNS). 
Now, with a full plate of acronyms, let's begin.


ADDIE provides a graphical user interface (GUI) to interact with the 
underlying data reduction software. ADDIE aims to guide the workflow to 
go from launching the reduction of raw neutron data to provided processed 
individual runs, post-processing of these individual runs  by applying 
optional corrections and summations, finally to visualization and 
output of the diffraction and pair distribution function data. 

ADDIE in pre-installed on \analysis at the SNS 
(\url{http://analysis.sns.gov}). Instructions are provided for Neutron Sciences users 
to setup the Remote Desktop capablities to view, analyze and download your data 
from anywhere you go. Options are provided Windows, Mac, and Linux. 
Also, contact support information is provided in the case of 
any issues or needed troubleshooting.

ADDIE is also avaible open-source. 
Please contact your Local Contact from NOMAD if you would like to 
know more about the repository (or to contribute!)
\section{Using ADDIE}

ADDIE development has been funded by the 
\href{https://www.energy.gov/}{US Department of Energy} (DOE). 

If you use ADDIE results in your published work, please cite the following papers:

(INSERT - ADDIE paper)
(INSERT - other reduction dependencies - Mantid, GUDRUN, etc.)

For any of the following features for your published work, please cite the associated papers:

(INSERT - specific feature papers)

From the following, you can download a \href{http://https://neutrons.ornl.gov/nomad}{BibTex file with all citations}.


