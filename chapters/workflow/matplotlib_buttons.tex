\section{Standard plot buttons}

For all plot views in ADDIE, you will see a common button set on the bottom left that is standard. Here, we describe what each button does. The buttons are shown below:

\noindent\makebox[\textwidth]{\includegraphics[width=0.5\paperwidth]{graphics/matplotlib_buttons.png}}

The buttons and their functions are as follows. You can also hover over these buttons to get a description of what they do:

\begin{itemize}

\item \guicmd{Home}: This allows you to reset the plot view to how it was initially when you loaded in the data. It is much like a "reset" button for the scale of the plot.

\item \guicmd{Back Arrow}: This will return to the previous view. If you have zoomed numerous times on the plot, this will take you to the last previous view.

\item \guicmd{Forward Arrow}: This will advance forward by one view. If you have zoomed numerous times on the plot and then went back, this will take you forward once more in your view.

\item \guicmd{Mutli-directional Arrows}: Enter Pan-mode. The left-click button will translate the figure in the plot view, the right-click will zoom the figure in the plot view \textbf{WARNING:} This conflicts with an option to manipulate the Legend. It can still work but typically we use the magnifying glass if we need to zoom the plot instead.

\item \guicmd{Magnifying Glass}: Allows you to box-select an area to zoom in on. Use the \guicmd{Back Arrow} and the {Home} buttons to return to previous views.

\item \guicmd{Subplot Config}: This allows you to adjust the plot view to span more or less of the whitespace in the plot.

\item \guicmd{Floppy Disk}: This allows you to save the plot as a static image in a few different file formats. These can be viewed external from ADDIE. On \analysis, you can use the \cmd{Eye of Mate} program that is available under the \cmd{Applications} drop-down on the top left.

\item \guicmd{check Box}: This allows you finer control over the figure options. You can change the curve color, add markers, change labels of figures in the plot, change the minimum and maximum values of each of the axes, and also change the scales from linear to log and vice versa. 

\end{itemize}